\section{Краткие теоретические сведения}
\subsection{Статистические ряды}
\subsubsection{Общие теоретические сведения}
При построении группированной выборки (интервального вариационного ряда) из псевдослучайной выборки $x{(1)} < x_{(2)} < \cdots < x_{(N)}$ число интервалов 
$[a_0, a_1], (a_1, a_2], \cdots, (a_{m-1}, a_m]$ определяется по формуле Стерджеса 
$
m = 1 + [\log_2 N]
$,
$a_0 = x_{(1)}$, $a_m = x_{(N)}, a_k - a_{k-1} = \frac{d}{m}$, $k = 1, \cdots, m$, где $d = a_m - a_0$. 

Интервальный ряд (группированная выборка) оформляется в виде:
\begin{tabular}{|c|c|c|}
\hline
    Интервалы & $n_i$ & $w_i$  \\
    \hline
    $[a_0, a_1]$ & $n_1$ & $w_1$\\
    \hline
    $(a_1, a_2]$ & $n_2$ & $w_2$\\
    \hline
    $\cdots$ & $\cdots$ & $\cdots$\\
    \hline
    $(a_{m-1}, a_m]$ & $n_m$ & $w_m$\\
    \hline 
    - & $\sum\limits_{i=1}^mn_i$ & $\sum\limits_{i=1}^mw_i$\\
    \hline
\end{tabular}

где $n_i$ - число значений, попавших в $i$-ый интервал; $w_i$ - относительная частота попадания в 
$i$-ый интервал, $w_i = \frac{n_i}{N}$.

\newpage

Ассоциированный статистический ряд:
    \begin{tabular}{|c|c|c|}
    \hline
        $x_i^*$ & $n_i$ & $w_i$  \\
        \hline
        $x_1^*$ & $n_1$ & $w_1$\\
        \hline
        $x_2^*$ & $n_2$ & $w_2$\\
        \hline
        $\cdots$ & $\cdots$ & $\cdots$\\
        \hline
        $x_m^*$ & $n_m$ & $w_m$\\
        \hline 
        - & $\sum\limits_{i=1}^mn_i$ & $\sum\limits_{i=1}^mw_i$\\
        \hline
    \end{tabular}


где $x_i^* = \frac{a_{i-1} + a_i}{2}$ - середина интервала $(a_{i-1}, a_i]$.
\subsubsection{Эмпирическая функция распределения}
\begin{equation*}
F_N^{\text{Э}}(x; x_1, x_2, \cdots, x_N)= \sum\limits_{x_k \leq x}\frac{1}{N} = 
\begin{cases}
0, &x<x_{(1)}\\
\frac{1}{N}, &x_{(1)} \leq x < x_{(2)} \\
\frac{2}{N}, &x_{(2)} \leq x < x_{(3)} \\
\frac{3}{N}, &x_{(3)} \leq x < x_{(4)} \\
\dots \\
1, & x \geq x_{(N)}
\end{cases}
\end{equation*}

\subsubsection{Выборочные оценки}
    \begin{tabular}{|l|l|}
    \hline
    Мат. ожидание& $\widetilde{a} = \frac{1}{N}\sum\limits_{i=1}^mn_ix_i^* = \sum\limits_{i=1}^m w_ix_i^*$\\
    \hline
    Дисперсия& $\widetilde{\sigma}^2 = \sum\limits_{i=1}^mw_i(x_i^*)^2 - \widetilde{a}^2$ \\
    \hline
    Выборочное значение хи-квадрат& $\chi^2_{B} = \sum\limits_{i=1}^m\frac{(n_i-Np_i^*)^2}{Np_i^*}$\\
    \hline
    \end{tabular}

где $x_{i}^* = \frac{a_{i-1} +a_i}{2}$ - середина интервала $(a_{i-1}, a_i]$.

\subsection{Показательное распределение}
    \begin{tabular}{|l|l|}
    \hline
    Плотность распределения& $\lambda e^{-\lambda x}$\\
    \hline
    Функция распределения& $1 - \lambda e^{-\lambda x}, x \geq 0$\\
    \hline
    Математическое ожидание&  $\lambda^{-1}$\\
    \hline
    Дисперсия&  $\lambda^{-2}$\\
    \hline
    Среднеквадратическое отклонение&  $\lambda^{-1}$\\
    \hline
    \end{tabular}


\subsection{Нормальное распределение}

    \begin{tabular}{|l|l|}
    \hline
    Плотность распределения& $\frac{1}{\sigma}\varphi(\frac{x-a}{\sigma}) = \frac{1}{\sigma\sqrt{2\pi}}e^{-\frac{(x-a)^2}{2\sigma^2}}$\\
    \hline
    Функция распределения& $\Phi(\frac{x-a}{\sigma}) = \frac{1}{\sigma\sqrt{2\pi}} \int_{-\infty}^{x}e^{-\frac{(t-a)^2}{2\sigma^2}}dt$\\
    \hline
    Математическое ожидание&  $a$\\
    \hline
    Дисперсия&  $\sigma^2$\\
    \hline
    Среднеквадратическое отклонение&  $\sigma$\\
    \hline
    \end{tabular}


\subsection{Равномерное распределение на отрезке $[a, b]$}
    \begin{tabular}{|l|l|}
    \hline
    Плотность распределения& $\frac{1}{b-a}, a \leq x \leq b$\\
    \hline
    Функция распределения& $\frac{x-a}{b-a}, a \leq x < b$\\
    \hline
    Математическое ожидание&  $\frac{a+b}{2}$\\
    \hline
    Дисперсия&  $\frac{(b-a)^2}{12}$\\
    \hline
    Среднеквадратическое отклонение&  $\frac{b-a}{2\sqrt{3}}$\\
    \hline
    \end{tabular}


\subsection{Общая схема проверки гипотез с помощью критерия $\chi^2$}

Найденное значение критерия $\chi^2_B$ сравнивается с критическим значением 
$\chi^2_{\text{кр}, \alpha}(l)$ из таблицы, где $\alpha$ - уровень значимости 
($\alpha = 0.05$), $l$ - число степеней свободы (
для \textbf{Задания 1} $l = m - 2$, 
для \textbf{Задания 2} $l = m - 3$, 
для  \textbf{Задания 3} $l = m - 1$, 
).

Если $\chi^2_{B} \leq \chi^2_{\text{кр}, \alpha}(l)$, то гипотеза о соответствии выборки распределению не противоречит экспериментальным данным (может быть принята) при уровне значимости $\alpha = 0.05$.

Если $\chi^2_{B} > \chi^2_{\text{кр}, \alpha}(l)$, то гипотеза о соответствии выборки распределению противоречит экспериментальным данным (не может быть принята) при уровне значимости $\alpha = 0.05$.

Таблица критических значений:
$$
\begin{array}{|c|c|c|c|c|c|}
    \hline
    l& 4 & 5 & 6 & 7 & 8 \\
    \hline
    \chi^2_{\text{кр}, \alpha}(l) & 9.5 & 11.1 & 12.6 & 14.1 & 15.5\\
    \hline
\end{array}
$$

\subsection{Общая схема проверки гипотез с помощью критерия Колмогорова}

Необходимо сравнить вычисленное значение $K_{N}=D_N \cdot \sqrt{N}$ с критическим значением $k_{\alpha}$ при уровне значимости $\alpha = 0.05$ и сделать вывод о справедливости гипотезы.\\
Где $D_N = \max\limits_{i \le j \le N} ( \max ( | F_N (x_{(j)}) - F (x_{(j)})|, 
| F_N ( x_{(j)} - 0 ) - F( x_{(j)} )  | ) ) $

Если $K_{N} \leq k_{\alpha}$, то гипотеза о соответствии выборки равномерному распределению на отрезке $[a,b]$ не противоречит экспериментальным данным (может быть принята) при уровне значимости $\alpha = 0.05$.

Если $K_{N} > k_{\alpha}$, то гипотеза о соответствии выборки равномерному распределению на отрезке $[a,b]$ противоречит экспериментальным данным (не может быть принята) при уровне значимости $\alpha = 0.05$.

Таблица критических значений:
$$
\begin{array}{|c|c|c|c|c|c|}
    \hline
    \alpha& 0.01 & 0.01 & 0.05 & 0.1 & 0.2 \\
    \hline
    k_{\alpha}& 1.63 & 1.57 & 1.36 & 1.22 & 1.07\\
    \hline
\end{array}
$$
\subsection{Средства языка программирования}%

Программа написана и выполнялась при использовании языка программирования Python 3.8.

\subsubsection{Округление с точностью до 5 знаков}
\begin{lstlisting}[language=Python]
import numpy as np
np.around(num, 5)
\end{lstlisting}

Для округления используется функция around библиотеки numpy языка программирования Python




%Функция для вычисления распределения нормального распределения:
%\begin{lstlisting}[language=Python]
%import scipy.stats as sps
%p = sps.norm.cdf(x, loc, scale)
%\end{lstlisting}
%
%Функция для нормального распределения представлена в коде выше. 
%Она использует функцию norm.cdf из библиотеки scipy.stats для получения функции распределения нормального распределения. 
%Данная функция принимает три аргумента:
%\begin{enumerate}
%    \item x - значение, для которого находится значение функции распределения.
%    \item loc - среднее выборочное значение ($\mu$).
%    \item scale - среднеквадратическое отклонение ($\sigma$).
%\end{enumerate}


\subsubsection{Функции плотности и функции распределения для нормального, экспоненциального и равномерного распределений}

\paragraph{Нормальное распределение}

Функция для вычисления распределения нормального распределения:
\begin{lstlisting}[language=Python]
import scipy.stats as sps
p = sps.norm.cdf(x, loc, scale)
\end{lstlisting}

Функция для нормального распределения представлена в коде выше. 
Она использует функцию norm.cdf из библиотеки scipy.stats для получения функции распределения нормального распределения. 
Данная функция принимает три аргумента:
\begin{enumerate}
    \item x - значение, для которого находится значение функции распределения.
    \item loc - среднее выборочное значение ($\mu$).
    \item scale - среднеквадратическое отклонение ($\sigma$).
\end{enumerate}

Функция для вычисления плотности вероятности нормального распределения:
\begin{lstlisting}[language=Python]
import scipy.stats as sps
p_pdf = sps.norm.pdf(x, loc, scale)
\end{lstlisting}

Функция norm.pdf из библиотеки scipy.stats вычисляет значение плотности вероятности нормального распределения. 
Функция принимает те же аргументы, что и norm.cdf.

%\paragraph{Экспоненциальное распределение}
%
%Функция для вычисления функции распределения экспоненциального распределения:
%\begin{lstlisting}[language=Python]
%import scipy.stats as sps
%p_cdf = sps.expon.cdf(x, loc, scale)
%\end{lstlisting}
%
%Функция для вычисления плотности вероятности экспоненциального распределения:
%\begin{lstlisting}[language=Python]
%import scipy.stats as sps
%p_pdf = sps.expon.pdf(x, loc, scale)
%\end{lstlisting}
%
%Функции expon.cdf и expon.pdf из библиотеки scipy.stats используются для вычисления функции распределения и 
%плотности вероятности экспоненциального распределения соответственно. Они принимают следующие аргументы:
%\begin{enumerate}
%    \item x - значение, для которого находится значение функции распределения или плотности вероятности.
%    \item loc - параметр сдвига ($\mu$).
%    \item scale - параметр масштабирования ($\lambda^{-1}$), где $\lambda$ - параметр интенсивности.
%\end{enumerate}
%
%\paragraph{Равномерное распределение}
%
%Функция для вычисления функции распределения равномерного распределения:
%\begin{lstlisting}[language=Python]
%import scipy.stats as sps
%p_cdf = sps.uniform.cdf(x, loc, scale)
%\end{lstlisting}
%
%Функция для вычисления плотности вероятности равномерного распределения:
%\begin{lstlisting}[language=Python]
%import scipy.stats as sps
%p_pdf = sps.uniform.pdf(x, loc, scale)
%\end{lstlisting}
%
%Функции uniform.cdf и uniform.pdf из библиотеки scipy.stats используются для вычисления функции распределения и 
%плотности вероятности равномерного распределения соответственно. Они принимают следующие аргументы:
%\begin{enumerate}
%    \item x - значение, для которого находится значение функции распределения или плотности вероятности.
%    \item loc начальная точка интервала ($a$).
%    \item scale - длина интервала ($b-a$), где $b$ - конечная точка интервала.
%\end{enumerate}




%Объект <<sps.norm>> принимает на вход параметры $a$, $\sigma$, определяющие распределение. 
%Возвращаемое значение - значении функции распределения нормального распределения с заданными параметрами в точке $x$.

%Функция для показательного распределения.
%\begin{lstlisting}[language=Python]
%import scipy.stats as sps
%p = sps.expon(0, 1/_lambda)
%\end{lstlisting}
%
%Объект <<sps.expon>> принимает на вход параметры сдвига ($0$) и масштаба ($1/\lambda$), определяющие распределение. 
%Возвращаемое значение - значении функции распределения показательного распределения с заданными параметрами в точке $x$.
%
%Функция для равномерного распределения.
%\begin{lstlisting}[language=Python]
%import scipy.stats as sps
%p = sps.uniform(a, b - a)
%\end{lstlisting}
%
%Объект <<sps.uniform>> принимает на вход начало отрезка ($a$) и его длину ($b-a$), определяющие распределение. 
%Возвращаемое значение - значении функции распределения равномерного распределения с заданными параметрами в точке $x$.


