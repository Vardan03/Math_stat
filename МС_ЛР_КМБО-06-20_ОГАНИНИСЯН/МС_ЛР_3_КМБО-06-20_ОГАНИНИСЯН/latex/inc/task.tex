\section{Условия}
\label{cha:tasks}

\subsection{Задание 1.} 
\textbf{Проверить} гипотезу о равенстве математических ожиданий с использованием распределения 
Стьюдента при уровне значимости $a=0,05$ 
для всех трёх пар наблюдаемых нормально распределенных случайных величин, 
выборки которых находятся в столбцах двухмерного массива 
$\{ u_{ij} | 1 \le i \le N, 1 \le j \le 3 \}$
    
\subsection{Задание 2.} 
\textbf{Проверить} с использованием однофакторного дисперсионного анализа
гипотезу о равенстве математических ожиданий при уровне значимости 0,05
трёх наблюдаемых нормально распределенных случайных величин, выборки
которых находятся в столбцах двумерного массива $\{ u_{ij} | 1 \le i \le N, 1 \le j \le 3 \}$

\subsection{Задание 3.}
\textbf{Проверить} гипотезу о равенстве математических ожиданий при уровне
значимости $a = 0,05$ для всех трёх пар наблюдаемых нормально
распределенных случайных величин, выборки которых находятся в столбцах
двумерного массива U , с помощью функций, в которых реализован t-критерий
Стьюдента.

\subsection{Задание 4.}
\textbf{Проверить} гипотезу о равенстве математических ожиданий при уровне
значимости $a=0,05$ для всех трёх пар наблюдаемых нормально
распределенных случайных величин, выборки которых находятся в столбцах
двумерного массива U , с помощью функций, в которых реализован t-критерий
Уэлча.


\subsection{Задание 5.}
\textbf{Проверить} гипотезу о равенстве математических ожиданий при уровне
значимости $a=0,05$ для трёх наблюдаемых нормально распределенных
случайных величин, выборки которых находятся в столбцах двумерного
массива U , с помощью функций, в которых реализован однофакторный
дисперсионный анализ.


\subsection{Задание 6.}
\textbf{Проверить} гипотезу о равенстве дисперсий при уровне значимости $a=0,05$
для всех трёх пар наблюдаемых нормально распределенных случайных
величин, выборки которых находятся в столбцах двумерного массива U , с
использованием распределения Фишера-Снедекора.


\subsection{Задание 7.}
\textbf{Проверить} гипотезу о равенстве дисперсий при уровне значимости $a=0,05$
для наблюдаемых нормально распределенных случайных величин,
выборки которых находятся в столбцах двумерного массива U , с помощью
функций, в которых реализован критерий Бартлетта.
